\subsection{Ερώτημα 1}
Έχω υλοποιήσει σε \texttt{python} ένα πρόγραμμα που διαβάζει 2 αρχεία:
το \texttt{a.csv} με έναν $n\times n$ πίνακα $\mathbf{A}$ και το \texttt{b.csv}
με έναν $1\times n$ πίνακα $\mathbf{b}$ και κάνει την $\mathbf{LU}$
ανάλυση του $\mathbf{A}$ και στην συνέχεια βρίσκει την λύση
στο σύστημα $\mathbf{Ax} = \mathbf{b}$

Αφού φτιάξει τους πίνακες $\mathbf{A}$ και $\mathbf{b}$
καλέι την συνάρτηση \texttt{gauss(A, b)} η οποία επιστρέφει το $x$.

Η συνάρτηση αρχικοποιεί τους πίνακες $\mathbf{L}$, $\mathbf{U}$ και
$\mathbf{x}$ και ξεκινάει υπολογίζοντας τους 2 πρώτους. Στην συνέχεια
βρίσκει το ενδιάμεσο $y$. Μετά βρίσκει το ανάποδο $x$ και το αντιστρέφει
για να επιστραφεί σωστά.

Το αρχείο \texttt{a.csv} πρέπει να περιέχει $n$ νούμερα ανα σειρά χωρισμένα με
\texttt{space} και $n$ σειρές ενώ το \texttt{b.csv} $n$ σειρές απο 1 νούμερο.

%%% Local Variables:
%%% mode: latex
%%% TeX-master: "../master"
%%% End:
