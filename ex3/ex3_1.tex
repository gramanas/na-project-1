\subsection{Ερώτημα 1}
Έχω υλοποιήσει σε \texttt{python} ένα πρόγραμμα που διαβάζει 2 αρχεία:
το \texttt{a.csv} με έναν $n\times n$ πίνακα $\mathbf{A}$ και το \texttt{b.csv} με έναν
$1\times n$ πίνακα $\mathbf{b}$ και κάνει την $\mathbf{LU}$ ανάλυση του $\mathbf{A}$
και στην συνέχεια βρίσκει την λύση στο σύστημα $\mathbf{Ax} = \mathbf{b}$

Διαβάζει τα αρχεία και φτιάχνει τους πίνακες $\mathbf{A}$ και $\mathbf{b}$
και ταυτόχρονα αρχικοποιεί τον πίνακα $\mathbf{L}$ σαν έναν $I_n$ και $\mathbf{U}$
όπου είναι ίδιος με τον $\mathbf{A}$ καθως είναι αυτός πάνω στον οποίο θα γίνουν
οι προσθαφαιρέσεις γραμμών αργότερα.

Στην συνέχεια βρίσκει τον τελικό πίνακα $\mathbf{U}$ και ταυτόχρονα συμπληρώνει
τον $\mathbf{L}$ με τους διαιρέτες. Μετά βρίσκει την λύση του συστήματος
$\mathbf{Ly} = \mathbf{b}$ και τέλος το $\mathbf{Ux} = \mathbf{y}$

Τυπώνει όλους τους πίνακες που χρειάστηκε καθώς και τον $\mathbf{A*x}$ για
επιβεβαίωση του σωστού αποτελέσματος.

Το αρχείο \texttt{a.csv} πρέπει να περιέχει $n$ νούμερα ανα σειρά χωρισμένα με
\texttt{space} και $n$ σειρές ενώ το \texttt{b.csv} $n$ σειρές απο 1 νούμερο.

%%% Local Variables:
%%% mode: latex
%%% TeX-master: "../master"
%%% End:
