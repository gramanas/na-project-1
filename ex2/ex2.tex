\section{Άσκηση 2}

Ο κώδικας σε αυτήν την άσκηση είναι παρόμοιος με τον κώδικα της πρώτης
με το μόνο άξιο σχολιασμού να είναι η τροποποιημένη μέθοδος της τέμνουσας:
\lstinputlisting[firstline=58, lastline=74]{ex2/ex2.py}

καθως η άσκηση λέει πως το $x_{n+3}$ αντικαθιστά το $x_n$
δημιουργώ έναν πίνακα 3 θέσεων και χρησιμοποιώ τον τελεστή $\%$
για να γίνεται η αντικατάσταση κυκλικά

\subsection{Ερώτημα 1}

Ακολουθούν τα αποτελέσματα με κατάλληλες αρχικοποιήσεις:
\begin{lstlisting}[language=C, mathescape=true]
$\dollar$ python ex2/ex2.py
++++ almost-Bisection ++++

Root in [0.80,0.90] after 25 loops: f(0.841067) = 0.000000

Root in [0.95,1.10] after 7 loops: f(1.047667) = 0.000000

Root in [2.30,2.80] after 25 loops: f(2.300524) = 0.000000

++++ almost-Newton - Raphson ++++

Starting at 0.80:
after 4 iterations the root is: f(0.841069) = 0.000000

Starting at 1.00:
after 6 iterations the root is: f(1.044162) = -0.000000

Starting at 2.50:
after 4 iterations the root is: f(2.300524) = -0.000000

++++ almost-Interpolation ++++

Starting points: [1.00, 2.00, 3.00]. After 10 iterations:
f(1.043381) = -0.000000

Starting points: [0.70, 0.80, 0.90]. After 10 iterations:
f(0.841069) = -0.000000

Starting points: [2.20, 2.30, 2.40]. After 4 iterations:
f(2.300524) = 0.000000
\end{lstlisting}

\subsection{Ερώτημα 2}
Εκτελώντας τον αλγόριθμο 10 φορές παρατηρώ οτι κάθε φορά συγκλίνει σε μια ρίζα
με διαφορετικό αριθμό επαναλήψεων, πράγμα αναμενόμενο καθως
διαλέγει την ρίζα τυχαία.
\begin{lstlisting}[language=C, mathescape=true]
0: Root in [0.00,3.00] after 32 loops: f(0.841069) = -0.000000
1: Root in [0.00,3.00] after 48 loops: f(2.300524) = 0.000000
2: Root in [0.00,3.00] after 28 loops: f(0.841069) = -0.000000
3: Root in [0.00,3.00] after 25 loops: f(2.300524) = 0.000000
4: Root in [0.00,3.00] after 34 loops: f(2.300524) = -0.000000
5: Root in [0.00,3.00] after 40 loops: f(2.300524) = -0.000000
6: Root in [0.00,3.00] after 48 loops: f(2.300524) = -0.000000
7: Root in [0.00,3.00] after 36 loops: f(2.300524) = -0.000000
8: Root in [0.00,3.00] after 19 loops: f(0.841070) = -0.000000
9: Root in [0.00,3.00] after 34 loops: f(2.300524) = 0.000000
\end{lstlisting}

\subsection{Ερώτημα 3}

Παρατηρώ πως η μέθοδος διχοτόμησης είναι σαφώς καλύτερη όταν επιλέγω το μεσαίο
σημείο αντί κάποιου τυχαίου.

Η μέθοδος N-R είναι παρόμοια χωρίς μεγάλες διαφορές και η μέθοδος της
τέμνουσας είναι καλύτερη στην εναλακτική της μορφή με τα 3 αρχικά σημεία
καθως γίνεται καλύτερη προσέγκιση με καμπύλη αντί μιας ευθείας.

Ακολουθούν αναλυτικά τα αποτελέσματα της σύγκρισης:
\begin{lstlisting}[language=C, mathescape=true]
++++ almost-Bisection ++++
Root in [0.80,0.90] after 18 loops: f(0.841069) = -0.000000
Root in [0.95,1.10] after 19 loops: f(1.047192) = 0.000000
Root in [2.30,2.80] after 20 loops: f(2.300524) = 0.000020

++++ almost-Bisection ++++
Root in [0.80,0.90] after 20 loops: f(0.841069) = -0.000000
Root in [0.95,1.10] after 8 loops: f(1.045441) = -0.000000
Root in [2.30,2.80] after 35 loops: f(2.300524) = -0.000000

++++ Newton - Raphson ++++
Starting at 0.80:
after 4 iterations the root is: f(0.841069) = 0.000000
Starting at 1.00:
after 6 iterations the root is: f(1.043596) = -0.000000
Starting at 2.50:
after 4 iterations the root is: f(2.300524) = -0.000000

++++ almost-Newton - Raphson ++++
Starting at 0.80:
after 4 iterations the root is: f(0.841069) = 0.000000
Starting at 1.00:
after 6 iterations the root is: f(1.044162) = -0.000000
Starting at 2.50:
after 4 iterations the root is: f(2.300524) = -0.000000

++++ Interpolation ++++
Starting points: [1.00, 3.00]. After 17 iterations:
f(1.043599) = -0.000000
Starting points: [0.70, 0.90]. After 31 iterations:
f(0.841069) = 0.000000
Starting points: [2.20, 2.40]. After 29 iterations:
f(13.610043) = -0.000000

++++ almost-Interpolation ++++
Starting points: [1.00, 2.00, 3.00]. After 10 iterations:
f(1.043381) = -0.000000
Starting points: [0.70, 0.80, 0.90]. After 10 iterations:
f(0.841069) = -0.000000
Starting points: [2.20, 2.30, 2.40]. After 4 iterations:
f(2.300524) = 0.000000
\end{lstlisting}

%%% Local Variables:
%%% mode: latex
%%% TeX-master: "../master"
%%% End: